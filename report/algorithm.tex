\section {实现内容}
\subsection {光线追踪}
这一部分在第二次小作业中已经基本实现,在本次实验中,对之前的光线追踪渲染器进行进一步修改及扩展。

\subsubsection {多线程下随机数生成}
在之前的作业中使用了C的srand方法种下种子,并调用rand方法来获取伪随机数。在多线程加速的情况下,会导致多个线程产生相同随机数的情况,但这并不是我们想要的结果。

在本次实验中,实现了RNGenerator类,以支持对多线程随机数的生成。RNGenerator类封装了C++11的随机数生成引擎,并且使用梅森旋转算法\footnote{\url{https://en.wikipedia.org/wiki/Mersenne_Twister}} 来进行随机数的生成。当然,最主要的目标还是要保证不同线程的种子不同。在对RNGenerator类进行实例化时,将会使用当时的时间及当前线程ID做异或操作,用上述结果作为随机数种子,从而保证了多线程随机数生成的随机性。

\subsubsection {额外渲染对象的支持}
实现了对四边形的支持,使用了[\cite{lagae2005efficient}]所述的光线与四边形快速求交的算法。以此为基础,即可实现对康奈尔盒模型的渲染。

\subsection {蒙特卡洛路径追踪}
路径追踪\footnote{\url{https://en.wikipedia.org/wiki/Path_tracing}}是一种对真实世界下全局光照的蒙特卡洛渲染方法,笼统来讲,该算法利用蒙特卡洛方法对渲染方程进行近似,从而达到对真实情况的逼近。在本次实验中,路径追踪渲染器用来作为直接/非直接光照的数据生成器及用来对比的Ground-Truth 数据生成器。

\subsubsection {渲染方程}
渲染方程\footnote{\url{https://en.wikipedia.org/wiki/Rendering_equation}} 于1986年被引入计算机图形学,在不考虑波长$\lambda$及时间$t$的情况下,其形式如下:
\[
    L_{o}(\bm{x}, \omega_{o})=L_{e}(\bm{x}, \omega_{0})+\int_{\Omega}f_{r}(\bm{x}, \omega_{i}, \omega_{o})L_{i}(\bm{x}, \omega_{i})(\omega_{i}\cdot\bm{n})d\omega_{i}
\]
其中\\
\begin{itemize}[noitemsep]
    \item $L_{o}(\bm{x}, \omega_{o})$ 代表在位置$\bm{x}$以及角度$\omega_{o}$的出射光
    \item $L_{e}(\bm{x}, \omega_{o})$ 代表上述位置及方向发出的光
    \item $\int_{\Omega} ...d\omega_{i}$ 代表入射方向半球的无穷小累加和
    \item $f_{r}(\bm{x}, \omega_{i}, \omega_{o})$ 是BRDF,也即在该点从入射方向到出射方向光的反射比例
    \item $L_{i}(\bm{x}, \omega_{i})$ 代表该点的入射光位置及方向$\omega_{i}$
\end{itemize}

\subsubsection {重要性采样}
在蒙特卡洛方法模拟积分时,采用重要性采样方法会使蒙特卡洛方法得到结果的方差减小,从而使算法更好更快地收敛至正确结果。

在路径追踪的过程中,当追踪光线与某物体存在一个交点时,需要随机确定一个反射方向从而继续追踪,此时就需要使用重要性采样来对不同的BRDF进行方向的选择。方向向量$\omega$可以由天顶角$\theta$及方位角$\phi$ 确定,接下来介绍两种BRDF的重要性采样方法。

如果该物体的BRDF是Lambertian,也即$f_{r}(\bm{x}, \omega_{i}, \omega_{o})=\frac{\rho}{\pi}$。假设$u,v$是两个属于$[0,1]$且均匀分布的随机数,那么
\[
    (\theta, \phi)=(arccos(\sqrt{1-u}), 2\pi v)
\]
概率分布函数为
\[
    pdf(\omega_{i})=\frac{cos(\theta)}{\pi}
\]

如果该物体的BRDF是Phong,也即$f_{r}(\bm{x}, \omega_{i}, \omega_{o})=\frac{k_{d}}{\pi}+k_{s}\frac{(n+2)}{2\pi}cos^{n}(\alpha)$。 由于表达式的前半部分与Lambertian一致,我们只讨论后半部分的情况。$u,v$的定义同上,则
\[
    (\theta, \phi)=(arccos(u^{\frac{1}{n+1}}), 2\pi v)
\]
概率分布函数为
\[
    pdf(\omega_{i})=\frac{n+1}{2\pi}cos^{n}(\alpha)
\]
其中$\alpha$为$\omega_{o}$与$\bm{n}$所成的夹角。

\subsubsection {直接光照\& 非直接光照}
由于在之后的实验中,会对场景中每个物体的直接光照及非直接光照进行训练,所以需要将当前的渲染方法拆成直接光照值和非直接光照值。

当追踪光线与某物体存在一个交点时,每次只需递归跟踪一条路径,现在追踪两条:一条指向光源$\omega_{direct}$,另一条与之前相同,仍然通过重要性采样随机进行跟踪$\omega_{indirect}$。其中,$\omega_{direct}$路径无需继续递归,只计算光源的发射值,$\omega_{indirect}$路径则继续递归跟踪下去。计算直接光照的伪代码见算法~\ref{alg:direct}:
\begin{algorithm}
\begin{algorithmic}
    \STATE $\omega_{i},pdf=$luminaireSample($x,\bm{n}$)
    \STATE $y=$traceRay($x,\omega_{i}$)
    \RETURN brdf($x,\omega_{i},\omega_{r}$)$\cdot$emittedRadiance($y,-\omega_{i}$)$/pdf$
\end{algorithmic}
\caption{directRadianceEst($x,\omega_{r}$)}
\label{alg:direct}
\end{algorithm}
这里提到了对光源的采样以及概率密度分布函数问题,这个问题将在下一小节说明。计算非直接光照的伪代码见算法~\ref{alg:indirect}:
\begin{algorithm}
\begin{algorithmic}
    \IF {random()$<$survivalProbability}
        \STATE $\omega_{i},pdf=$brdfSample($x,\bm{n}$)
        \STATE $y=$traceRay($x,\omega_{i}$)
        \RETURN brdf($x,\omega_{i},\omega_{r}$)$\cdot$reflectedRadianceEst($y,-\omega_{i}$)$/(pdf\cdot$survivalProbability$)$
    \ELSE
        \RETURN 0
    \ENDIF
\end{algorithmic}
\caption{indirectRadianceEst($x,\omega_{r}$)}
\label{alg:indirect}
\end{algorithm}
 